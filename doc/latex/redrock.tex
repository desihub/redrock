\title{Redrock: Redshift fitting spectra with different resolutions}
\author{Stephen Bailey, David Schlegel}
\date{\today}

\documentclass[12pt]{article}

\begin{document}
\maketitle

\abstract{Redrock is a redshift fitting toolkit for multiple spectra of
the same object, each of which can have a different wavelength grid and
resolution.  As input, redrock takes spectra defined by their flux,
uncorrelated errors, and resolution matrix as calculated by the
spectroperfectionism algorithm.}

%-------------------------------------------------------------------------
\section{Inputs}

\begin{enumerate}
    \item Multiple flux calibrated spectra $\mathbf{s}_i$ with uncorrelated
        pixels $s_{i\lambda}$.
    \item Uncertainties $\sigma_{i\lambda}$ on $s_{i\lambda}$; also expressed
        as a diagonal covariance matrix $C_i = \mathrm{diag}(\sigma_i^2)$
    \item A perfect resolution restframe template $\mathbf{t}_0$ evaluated at
        redshift $z$ is $\mathbf{t}_z$, i.e. $\mathbf{t}_z(\lambda) =
        \mathbf{t}_0(\lambda(1+z))$
    \item $P_i(\mathbf{t}_z)$ resamples $\mathbf{t}_z$ onto the same
        wavelength pixelization as $\mathbf{s}_i$
    \item Resolution matrix $R_i$ that convolves $P_i(\mathbf{t}_z)$ to the
        resolution of spectrum $\mathbf{s}_i$
\end{enumerate}

Notes:
\begin{enumerate}
    \item The pixels $s_{i\lambda}$ and $[P_i(\mathbf{t}_z)]_\lambda$
        are a flux density interpreted as the
        integral of the spectrum over each pixel divided by
        the pixel width; it is not just an evaluation of the continuous
        flux density at a single wavelength.  When used as a subscript,
        $\lambda$ is a pixel index representing the center
        of the pixel.
\end{enumerate}

%-------------------------------------------------------------------------
\section{Evaluating $\chi^2$}

The $\chi^2$ for a given template evaluated at redshift $z$ compared to the
spectra $\mathbf{s}$ is then
\begin{equation}
    \chi^2_z = \sum_i \sum_\lambda
        \left( {s_{i\lambda} - a [R_i P_i(\mathbf{t}_{z})]_\lambda) \over
            \sigma_{i\lambda}} \right)^2
\end{equation}
where $a$ is a flux normalization term applied to the template to match the
data.
Expressed in matrix notation this is
\begin{eqnarray}
    \Delta_i & = & \mathbf{s}_i - a R_i P_i(\mathbf{t}_z) \\
    \chi^2_z & = & \sum_i \Delta_i^T C_i^{-1} \Delta_i
\end{eqnarray}

%-------------------------------------------------------------------------
\section{Flux correction terms}

In practice, broadband errors in the flux calibration of $\mathbf{s}_i$
should be absorbed as individual $a_i(\lambda)$ functions rather than a
single flux calibration term $a$.  {\it e.g.}~$a_i(\lambda)$ may be expressed
as a sum of Legendre polynomials $L_j(\lambda)$:
\begin{eqnarray}
    a_i(\lambda) & = & \sum_j b_{ij} L_j(\lambda) \\
    \mathbf{a}_i & = & L \mathbf{b}_i
\end{eqnarray}
In this case, the $\chi^2_z$ becomes
\begin{eqnarray}
    \Delta_i & = & \mathbf{s}_i - \mathbf{a}_i \odot R_i P_i(\mathbf{t}_z) \\
    \Delta_i & = & \mathbf{s}_i - (L \mathbf{b}_i) \odot (R_i P_i(\mathbf{t}_z)) \\
    \chi^2_z & = & \sum_i \Delta_i^T C_i^{-1} \Delta_i
\end{eqnarray}
Where the notation $\mathbf{x} \odot \mathbf{y}$ indicates the elementwise
multiplication of vectors $\mathbf{x}$ and $\mathbf{y}$.  The vectors
$\mathbf{b}_i$ are coefficients to solve to minimize the $\chi^2_z$.

\textbf{Question}: is there any actual benefit to fitting for fluxcorr terms
during template fitting, or should we just consider those to be solved
separately ahead of time to keep the template fitting nice and simple?

%-------------------------------------------------------------------------
\section{Pixelizing the template}

$P_i(\mathbf{t}_z)$ pixelizes the continuous function $\mathbf{t}_z(\lambda)$
by integrating over each pixel.  In practice, $\mathbf{t}_z$ is reported as
discrete samplings of the function at wavelengths $\lambda$.

TODO: define exactly how to integrate those pixels.
The current implementation is just trapezoidal integration.

%-------------------------------------------------------------------------
\section{Optimizations}

Spectroperfectionism extractions can be performed on a common wavelength
grid for multiple spectra on the same camera, even if they don't perfectly
align with the CCD pixels.  Thus $P_i(\mathbf{t}_z)$ is not unique for every
spectrum $i$.  It can be pre-calculated for each camera and each redshift $z$
and then applied to many spectra.  In the previous equations $P_i$ is treated
as different for every spectrum to be completely general, but in practice
this is not necessary.

\end{document}
